\documentclass[a4paper]{article}
\usepackage[utf8]{inputenc}
\usepackage[english]{babel}
\usepackage[a4paper]{geometry}
\geometry{hmargin=2cm, vmargin = 3cm}
\usepackage{microtype}
\usepackage{graphicx}
\graphicspath{ {images/} }
\usepackage{caption}
\usepackage{subcaption}
\usepackage{amssymb,amsmath,amsthm}
\usepackage{bbm}
\usepackage[ruled]{algorithm2e}
\usepackage{acronym}
\usepackage[colorlinks=true, allcolors=blue]{hyperref}
\usepackage{cleveref}
\crefname{algocf}{alg.}{algs.}
\Crefname{algocf}{Algorithm}{Algorithms}
\usepackage{amsmath}
\usepackage{floatrow}
\usepackage[font=bf]{caption}
\usepackage{wrapfig}

%\setlength{\parindent}{0pt}

\newacro{TFIDF}{Term Frequency - Inverse Document Frequency}
\newacro{LDA}{Latent Dirichlet Allocation}
\newacro{NMF}{Non-negative Matrix Factorization}
\newacro{SVD}{Singular Value Decomposition}
\newacro{LSML}{Least Square-residual Metric Learning}
\newacro{AUC}{Area Under Curve}
\newacro{PCA}{Principal Component Analysis}

\author{Arthur Imbert}
\title{Integration of heterogeneous datasets}
\date{October 2017}

\begin{document}
	
	\maketitle
	
	The internship took place in the \href{https://team.inria.fr/parietal/}{Parietal} team at INRIA. My supervisors was Ga\"el Varoquaux.
	
	\section{Summary}
	
	\section{Introduction}
	
	\subsection{Heterogeneous data, producers and value}
	
	Nowadays, the production of data has considerably grown. Each organization exploits and issues data, using its own schema. Most of the time, these files are produced for specific purpose, taking into account some internal rules or constraints. That leads to a vast amount of data available online, often sharing the same file formats, but still deeply heterogeneous by their structure. Indeed, this heterogeneity complicates the integration of different files.
	
	By the same time, the current enthusiasm around data science and its predictive models shows a high potential for crossing data. 
	
	\subsection{A theoritical framework}
	
	\paragraph{Related problems}
	
	\paragraph{Our strategy}
	
	\section{An application to Open Data}
	
	\subsection{Open and heterogeneous data}
	
	In order to collect a vast amount of heterogeneous data we exploit the Open Data available online.
	
	% definition of Open Data 
	
	A French platform exists, gathering all the information needed to use the files issued by the French public organizations: \href{http://www.data.gouv.fr/fr/}{data.gouv.fr}. This website hosts and indexes multiple sources
	
	\subsection{How can we reuse the data}
	
	%\begin{figure}
	%	\centering
	%	\includegraphics[width = .9\linewidth]{}
	%	\caption{Neighborhood examples}
	%	\label{fig:neighbor-police}
	%\end{figure}
	
	\subsection{An academic angle}
	
	\subsection{Internship's goal}
	
	\section{From a bunch of heterogeneous data to a recommendation system}
	
	Before any analyze, we build our dataset. From the platform \href{http://www.data.gouv.fr/fr/}{data.gouv.fr}, we collect quantities of metadata. It generally includes an URL address where the file is stored. We then directly download as many files as possible. For each file downloaded, we try to clean it such as we can display its content in a tabular form.
	
	At that point, we start the analysis. Firstly by preprocessing files' textual content. The result consists in a \ac{TFIDF} matrix. It allows us to weight word occurrences within our corpus according to their importance for each file. Secondly by building a vectorial topic space to represent every files (a file embedding) and based on the \ac{TFIDF} matrix. This also includes an optimization step where we compare two topic extraction algorithms - \ac{NMF} and \ac{SVD} - along with different configurations. Thirdly by looking for closest neighbors in this "topic space", in order to interpret distances within the embedding. Using a pretext task (the suggestion of a reuse between pairs of files) and a metric learning algorithm, \ac{LSML}, we also try to strengthen our file embedding and the interpretation of its clusters.
	
	%\begin{figure}
	%	\centering
	%	\includegraphics[width = .9\linewidth]{}
	%	\caption{Overview of the process}
	%	\label{fig:global-pipeline}
	%\end{figure}
	
	\subsection{Build a database}
	
	The first step is to store a sufficient amount of dirty data in order to keep a statistically significant volume of data throughout all our process.
	
	\paragraph{Collect metadata}
	
	As described above, the platform \href{http://www.data.gouv.fr/fr/}{data.gouv.fr} offers us the opportunity to easily collect a numerous files from different French public organizations. A RESTful API exists in the website. It allows us to get, from a HTTP request, several metadata about more than 25000 pages. These include producer's identity, a description of the page's content, some related tags, the different files belonging to the page and, the most important, an URL for each file. 
	
	Some URL lead us to another website which host files, some directly launch the download of file. We only focus on the latters. It already represents around 70000 URL.
	
	\paragraph{Download data}
	
	We use DRAGO, an INRIA's server, and a parallelized process to efficiently download as many files as possible. We use DRAGOSTORAGE, another server, to store the results of our downloads. By the end it represents around 350 Gb of data for more than 55000 files.
	
	\subsection{Clean and filter data}
	
	Among these files, there are several format: text, JSON, XML, spreadsheet, zip... Indeed, the second step consists in inferring the nature of files in order to filter them. The right process could then be applied in a attempt to clean files and reshape their content in a tabular form.
	
	%\begin{figure}
	%	\centering
	%	\begin{subfigure}{0.50\textwidth}
	%		\centering
	%		\includegraphics[width = .9\linewidth]{}
	%		\caption{An example of file easy to clean}
	%		\label{fig:good-table}
	%	\end{subfigure}
	%	\begin{subfigure}{0.50\textwidth}
	%		\centering
	%		\includegraphics[width = .9\linewidth]{}
	%		\caption{An example of 'dirty data'}
	%		\label{fig:bad-table}
	%	\end{subfigure}
	%	\caption{Cleaning data}
	%	\label{fig:example-table}
	%\end{figure}
	
	\paragraph{Filter the formats}
	
	According to the supposed format of the file, we apply a different strategy to clean it. That involves to avoid the empty files and detect its MIME type (using \emph{magic} library). If we meet a zip file, we repeat the procedure for every zipped items. We mainly focus on three types of documents:
	\begin{itemize}
		\item Text files, from which we try to detect a potential tabular form like in a CSV file.
		\item Spreadsheets, generally with an obvious tabular form that could make the inference of irregularities more difficult. 
		\item Semi-structured files, generally containing geographical data like in a GEOJSON, easy to read, but difficult to flatten within a table.
	\end{itemize}
	
	\paragraph{CSV files}
	
	For every file detected as text file, the very first operation executed is a test to determine if it could be a semi-structured file (JSON or GEOJSON). This test simply consists in reading the file, using the loaders from \emph{json} and \emph{geopandas} libraries. If an error occurs, the test fails and we process the file as a proper text file and a potential CSV or TSV file.
	
	Several operations are executed in order to extract some data in a tabular form. Using \emph{chardet} library, the encoding is inferred. If the latter is still undetermined, we read the file by default with an "utf-8" encoding. If the file is long enough, a sample of rows is randomly extracted to speed up the next steps. Using \emph{Sniffer} class from \emph{csv} library, we try to detect a column delimiter. For example, in a CSV file, it's a comma. This is a character which should occur the same number of times on each row. \emph{Sniffer} doesn't use a "all or nothing" approach and then allows some irregularities in the data. Once we have a delimiter, the number of columns can be easily computed. In order to avoid rows with commentary or titles, we only keep those with the right number of columns (or delimiter occurrences). The last step, and the more difficult, before saving data within a \emph{pandas} Dataframe, consists in inferring a pertinent header row. We test the consistency of data types for each row, excepted the targeted one (which should be a string). If every columns nearly keep the same data type along file, we consider the targeted row as a potential header. The operation is repeated for the first rows until a proper header row is found. If nothing return, values by default are assigned as column names. 
	
	%\begin{figure}
	%	\centering
	%	\includegraphics[width = .9\linewidth]{}
	%	\caption{CSV example}
	%	\label{fig:csv-example}
	%\end{figure}
	
	\paragraph{Excel files}
	
	When a file is detected as a spreadsheet, it is read and edited using \emph{xlrd} and \emph{xlutils} libraries, respectively. The main difficulty is to skip the wrong rows (title, commentaries) and clearly delimited the pertinent data. Contrary to the previous cases, we can't infer this information based on a wrong number of delimiter: in a spreadsheet, the tabular format is already given. Rows have the same number of columns. But sometimes a row has too many empty cells or the latter are merged. So, before parsing the data through a dataframe and looking for the header, we need to preprocess the cells.
	
	Cell by cell, we first replace characters that could mistake us while reading the document as a text file (for example "\textbackslash{n}" and "\textbackslash{r}"). Then, we fill merged cells by simply duplicating the information. We collect the potential name of its different sheets. For each sheet, we avoid the rows with an unexpected number of empty cells. The sheet is then parsed within a \emph{pandas} Dataframe and saved as a text file. That allows us to apply the same operation executed with the CSV files in order to infer a header row. By the end, a single spreadsheet should return one cleaned table per sheet.
	
	%\begin{figure}
	%	\centering
	%	\includegraphics[width = .9\linewidth]{}
	%	\caption{Excel example}
	%	\label{fig:excel-example}
	%\end{figure}
	
	\paragraph{JSON, GEOJSON and XML files}
	
	When a text file is detected, as previously explained, we try to read it as a JSON, then as a GEOJSON. The same process is then applied for both of them. Concerning the XML files, we just load it as a JSON file before, using the \emph{xmljson} library. 
	
	All these files can be read as a tree. The idea is to recursively explore the tree structure of the file in order to find a pertinent branch to flatten in a dataframe. The pattern we are looking for is a list of dictionaries, where each item of the list is an observation and the dictionaries' keys are features. For performance reasons we just explore the first layers of the tree. By the end, we flatten the part of the tree with the largest number of items and a consistent number of features. This involves \emph{json\_normalize} function from \emph{pandas} library. Generally, the header automatically inferred is pertinent.
	
	%\begin{figure}
	%	\centering
	%	\includegraphics[width = .9\linewidth]{}
	%	\caption{JSON example}
	%	\label{fig:json-example}
	%\end{figure}
	
	\paragraph{Errors and extradata}
	
	Through our cleaning process, numerous files are skipped because of their detected MIME type (PDF, images, etc.) or due to an error. For those which return a cleaned table, we also collect extradata. This are parts of the original file we don't include in the final table. While cleaning the text and Excel files, we still save separately rows with the wrong number of columns. It often represents a commentary, a source or the title. With the semi-structured files, we keep the part of the tree we don't flatten as extradata.
	
	\subsection{Build a file embedding}
	
	Once we have enough tables, with extradata for some of them, the third step consist in building a file embedding. That means we want to represent our files within a vectorial space in order to easily compute distances between files, and group them by cluster. We define a pretext task, predict a reuse between two files, to validate our process and determine if our embedding represent the corpus efficiently: two files reused together should be close in our space. 
	
	\paragraph{Text data preprocessing}
	
	We mainly use the textual content of the files to determine if it worths to reused them together or not. Thus a preprocessing step is decisive. As every cleaned table is saved as a text file, the goal is to adequately count and weight words occurrences. 
	
	We distinguish three parts for a cleaned file: the content of table, its header row and some extradata (in a separate file). Thus, the first time-consuming task includes creating three different $N_{files} \times N_{words}$ occurrence count matrices with $N_{files}$ the number of files and $N_{words}$ the number of different words. This tokenization process is executed with \emph{CountVectorizer} function of \emph{sklearn} library. For each matrix, we normalize its rows, so that the sum of each row returns the same value. This penalizes large files with a lot of textual content. Normalizing the three matrices separately also emphasizes header and extradata, two parts with generally a lower but nonetheless quite pertinent textual content. Ultimately we mix the three matrices such that,
	
	\begin{equation}
		D = 0.5 * Content + 0.25 * Header + 0.25 * Extradata		
	\end{equation}
	
	The next task is a stemming algorithm from \emph{nltk} library. During this process, we reduce inflected words to their root form (or stem). For example, such cuts could happen,
	
	\[
		different \rightarrow differ
	\]
	\[
		differ \rightarrow differ
	\]
	\[
		differently \rightarrow differ
	\]

	That makes our occurrence count matrix $D$ more consistent with spelling mistakes, grammatical changes and conjugation. During the stemming process we keep an history of changes in order to determine, for each stem, the most frequent inflected word. We can then substitute the stem by its most frequent origin in order to have a list of existing words as a result. In the previous example, our matrix $D$ could have a columns $different$ instead of $differ$. This is called a lemmatization process.
	
	For the last preprocessing task we compute a \ac{TFIDF} matrix from $D$. On the one hand, some words are frequent in every file (for example "the", "a", "is"). They usually appear to be less informative. Some of these words could even be regarded as noise and directly removed from the matrix $D$, using a predefined \emph{stop words} list. On the other hand, rarer terms may often be more interesting and would deserve a bigger weight. In order to emphasize them, we use \emph{TfidfTransformer} function from \emph{sklearn} library. This transformation weights each word frequency by its inverse document frequency. We have
	
	\begin{equation}
		TFIDF(t, d) = TF(t, d) \times IDF(t)
	\end{equation}
	
	with 
	
	\begin{equation}
		IDF(t) = log(\frac{N_{files}}{1 + DF(t)})
	\end{equation}
	
	where term-frequency $TF(t, d)$ is the number of occurrences of term $t$ in document $d$ and document-frequency $DF(t)$ is the number of documents that contain term $t$. Each row is then normalize with a $L_{2}$ norm. Ultimately, the most discriminant words for a specific file (or row) have a higher floating score. 
	
	\paragraph{Topic modeling}
	
	At that point, with \ac{TFIDF} matrix we already have a representation of files within a very high-dimensional vectorial space. Topic modeling can be defined as a dimensionality reduction technique. We use it to reveal a hidden semantic structure in our corpus. We then expect that the topics produced are clusters of similar words. After dimensionality reduction, a file should be interpreted as a mixture of topics. Our aim is to compute a $N_{files} \times N_{topics}$ matrix $W$, with $N_{topics}$ the number of topics (in our case, an integer to choose between 5 and 100). During the internship we used two different topic modeling algorithms: \ac{NMF} and \ac{SVD}. They basically factorize the \ac{TFIDF} matrix into the product of $W$, which represents files within topic space, and a $N_{topics} \times N_{words}$ matrix $H$, which represents topics within words space.
	
	\paragraph{\acf{NMF}}
		
	\ac{NMF} can be used if data matrix doesn't contain negative values. Indeed, it perfectly fit with textual data and more precisely occurrence count. It's also widely used in document clustering, recommender systems or computer vision. 
	
	%\begin{figure}
	%	\centering
	%	\includegraphics[width = .9\linewidth]{}
	%	\caption{NMF}
	%	\label{fig:nmf}
	%\end{figure}

	Let $V$ be our \ac{TFIDF} matrix. The algorithm minimizing a distance between $V$ and the matrix product $W.H$, with $W$ and $H$ containing non-negative elements. We use \emph{sklearn}'s implementation where the distance minimized is a squared Frobenius norm, such as
	
	\begin{equation}
		d_{\mathrm{Fro}}(A, B) = \frac{1}{2} ||A - B||_{\mathrm{Fro}}^2 = \frac{1}{2} \sum_{i,j} (A_{ij} - {B}_{ij})^2
	\end{equation}
		
	The objective function to minimize is
	
	\begin{equation}
	\frac{1}{2} ||V - WH||_{\mathrm{Fro}}^2 + \alpha\lambda||W||_{1} + \alpha\lambda||H||_{1} + \frac{1}{2}\alpha(1 - \lambda)||W||_{\mathrm{Fro}}^2 + \frac{1}{2}\alpha(1 - \lambda)||H||_{\mathrm{Fro}}^2
	\end{equation}
	
	with $\alpha$ a constant that multiplies the regularization terms, and so controls the sparsity of $W$ and $H$, and $\lambda$ a constant to balance $L_{1}$ and $L_{2}$ elementwise penalizations. Ultimately, we have
	
	\begin{equation}
		V \approx W.H
	\end{equation}
	
	under the constraints $W \succeq 0$ and $H \succeq 0$.
	
	\paragraph{\acf{SVD}}
	
	It approaches a spectral decomposition. We use \emph{sklearn}'s implementation which is a truncated version (called Truncated \ac{SVD}). We approximate V such as
	
	\begin{equation}
		V \approx V_k = U_k \Sigma_k P_k^\top
	\end{equation}
	
	The diagonal entries $\sigma_{i}$ of $\Sigma$ are $M$'s singular values. The columns of $U$ and $V$ are called the left-singular vectors and right-singular vectors of $M$, respectively. This implementation is called "truncated" because we only compute the $k$ column vectors of $U$ and row vectors of $P$ corresponding to the $k$ largest singular values $\Sigma$. The topic space is defined by $U_k \Sigma_k^\top$, with $k$ features (or topics). In relation with our framework, we have
	
	\[
		k = N_{topics}
	\]
	
	\[
		W = U_k.\Sigma_k
	\]
	
	\[
		H = P_k^\top
	\]
	
	Finally, as this algorithm doesn't center the data before computing the singular value decomposition, it's still efficient with sparse matrix from \emph{scipy} library. That perfectly fits with our \ac{TFIDF} matrix.
	
	\paragraph{Metric learning}
	
	In practice, the previous unsupervised step is enough to build an efficient file embedding. For a certain $N_{topics}$, words clusters could form homogeneous topics which are even interpretable. Any Euclidean distance between two files would be synonymous of similarity. Two files close in $W$ would have the same semantic structure, the same textual content. This is a first step toward reuse prediction. Indeed, it's pertinent to reuse similar files in order to increase data volume. Our goal is to go further in the analysis, to see if metric learning could yet improve the way we interpret distances. We want to base reuse prediction on more than just similarity. With a supervised metric learning algorithm, the idea is to learn from the existing reuses the potential of crossing two files weighted in two strictly different, but yet complementary, topics.
	
	Let's consider two files represented in a topic space by two vectors $x$ and $y$. A distance can be computed between them, using for example the quadratic norm
	
	\begin{align}
	\begin{split}
	||x - y||^2_2 &= \sum_{i=1}^{N_{topics}}(x_i - y_i)^2\\ 
				  &= (x - y)^T(x - y)
	\end{split}
	\end{align}
	
	However, we can assign a different weight for each dimension, such that
	
	\begin{align}
	\begin{split}
	||x - y||^2_w &= \sum_{i=1}^{N_{topics}}w_i\times(x_i - y_i)^2, \quad w_i \geq 0 \: \forall i\\
			 	  &= (x - y)^Tdiag(w)(x - y)
	\end{split}
	\end{align}
	
	With a more generally form we have
	
	\begin{align}
	\begin{split}
	||x - y||^2 = (x - y)^TM(x - y), \quad M \succeq 0
	\end{split}
	\end{align}
	
	For $z = x - y$ and we define $\tilde{z} = M^{\frac{1}{2}}z$, so
	
	\begin{align}
	\begin{split}
	||\tilde{z}||^2_2 &= z^T(M^{\frac{1}{2}})^TM^{\frac{1}{2}}z\\
					  &= z^TMz\\
					  &= ||z||^2
	\end{split}
	\end{align}
	
	The matrix $M$ is called Mahalanobis matrix, such that $M = L^TL$ with $L = M^{\frac{1}{2}}$. The aim of a metric learning algorithm is to compute $M$ and its factor $L$. Thus we can apply the latter as a transformation matrix to convert any multidimensional difference between two files $(x - y)$. Standard Euclidean distances can then be easily computed from the new transformed vector $L(x - y)$. It amounts to weight each feature (or dimension) and combination of features.
	
	One big advantage of this method is the possibility to use a supervised framework. Thus, the Mahalanobis matrix is computed such that the transformation matrix $L$ moves reused files closer within our learned metric space. To do that, we use a \ac{LSML} algorithm from the \emph{metric\_learn} library.
	
	\paragraph{\acf{LSML}}

	This algorithm uses pairs' label to learned the Mahalanobis matrix $M$ from constraints. The latters are defined as relative comparisons. For example, the distance between two reused files A and B is supposed to be smaller than the distance between two non reused files C and D. Contrary to a usual classification problem, we don't want to class a pair as reusable or not, as this objective seems meaningless. The results we are looking for are only relative.
	
	Let $x_i \in \mathcal{R}^{N_{topics}}$, $\forall i$, we define a set of relative comparison as
	
	\begin{equation}
		\mathcal{C} = \{(x_{a}, x_{b}, x_{c}, x_{d}):d(x_{a}, x_{b}) < d(x_{c}, x_{d})\}
	\end{equation}
	
	and the distance function as a $M$-transformed quadratic norm
	
	\begin{equation}
		d_{M}(x_{i}, x_{j}) = \sqrt{(x_{i} - x_{j})^{T}M(x_{i} - x_{j})}
	\end{equation}
	
	with $M$ the Mahalanobis matrix we want to learn. We also define a Hinge loss
	
	\begin{equation}
		L(d(x_{a}, x_{b}) < d(x_{c}, x_{d})) = H(d_{M}(x_{a}, x_{b}) - d_{M}(x_{c}, x_{d}))
	\end{equation}
	
	where a Hinge function is defined as
	
	\begin{equation}
		H(x) = \begin{cases}
				0 & \mbox{if } x \leq 0 \\
				x^{2} & \mbox{if } x > 0
			   \end{cases}
	\end{equation}
	
	The loss function is zero when the constraint $\mathcal{C}$ is satisfied, otherwise it's the difference of the new learned distance (the Euclidean distance in the $L$-transformed space). This loss penalizes any learned metric which moves reused pairs away within the vectorial space (and breaks the relative constraint). We also note that the shape of the Hinge function doesn't penalize the matrix $M$ for moving reused pairs too close. Thus the closeness between reused files may be excessive sometimes.
	
	Once a proper Mahalanobis matrix is learned from a set of pairs file (reused and not reused), we can directly transform the file embedding $W$ with the transformation matrix $L$. It modifies the weigh of each file through the topics and so its vectorial representation within the topic space.
	
	\subsection{Find neighborhood}
	
	Once we have a file embedding where distances can be interpreted as a similarity score between files and potential reuse, the fourth step consists in implementing neighbor searches. We look for pertinent groups of files in order to validate the distances between actual reused files. Yet, it would allow us to find files which belong to a cluster of reused files, but no one reused yet.
	
	We use the \emph{NearestNeighbors} function from the \emph{sklearn} library to implement neighbor searches. We don't specify the number of neighbors we want. It relies on the local density of points. It's a radius-based neighborhood. Every file within a fixed radius from the targeted file is counted as neighbor. We choose a basic L2 distance, since we already transform our topic space with our transformation matrix $L$. The critical parameter of this step is the radius. Too small, very few neighbors would be assigned to our target. Too big, the algorithm would class noisy files as neighbors. The choice of a pertinent radius is detailed in the next section, along with the optimization of several other parameters.
	
	\section{Results and validation}
	
	We first present here the dataset built at the end of downloading and cleaning steps. Then, we explain how several actions can influence our final results and our file embedding. The pipeline includes different parameters we need to set up. From the \ac{TFIDF} matrix, a topic extraction model and the number of topics extracted have to be specified in order to perform a dimensionality reduction. The metric learning process has to be evaluated too, along with the way we compute an affinity score between two files. Hence, we define a cross-validation framework to test different settings and evaluate their pertinence regarding our objective: suggest potential reuses. Once we find our optimal parametrization, the best embedding is computed and its related results can be displayed and interpreted.
	
	\subsection{Downloaded and cleaned files}

	\paragraph{Metadata}
	
	From the platform \href{http://www.data.gouv.fr/fr/}{data.gouv.fr} we collect the metadata of 26856 different pages. Each page comes from an unique producer and deals with a specific topic. Usually, one page hosts several files. Relevant metadata can be collected at the page level, such that associated tags, localization of the data and the most important, the potential reuses of this page. We do not know precisely what are the files from the page that have been reused. So, by definition, every files from the same page present the same information about reuses.
	
	\paragraph{Downloaded files}
	
	We try to collect as many files as possible. There are 96629 different URL, one per file. However, an URL can lead to a different website (the producer's one) and it requires its own API or scraping algorithm to be downloaded. That is why, we focus on the files we can directly download from \href{http://www.data.gouv.fr/fr/}{data.gouv.fr}, with a request on the URL. There are potentially 71368 available files. During the downloading process, we manage to load 55138 files and store 348GB in INRIA's server named DRAGOSTORAGE. It represents 17798 different pages. The main extensions downloaded are ZIP files (30345) and text files (19096). Other extensions type represent a thousand of occurrences or less (spreadsheets, PDF, XML, HTML, Powerpoint, images, etc.). Almost 99\% of the 16230 failures are \emph{HTTPError}. The main errors are \emph{Error 404: Not Found}, \emph{Error 500: Erreur Interne de Servlet}, \emph{Error 504: Gateway Time-out} or \emph{Error 410: Gone}. The vast majority concerns shapefile (6344), JSON files (5975) or ZIP files (3698). Most of the time, as we fail to download the file because of external reasons, we can hardly improve our rate of success. Usually, 
	
	\paragraph{Cleaned files}
	
	\begin{figure}[]
		\minipage{0.48\textwidth}
		\includegraphics[width=\linewidth]{images/"extension".pdf}
		\label{fig:Original extension}
		\subcaption{Extensions}
		\endminipage\hfill
		\minipage{0.48\textwidth}
		\includegraphics[width=\linewidth]{images/"size".pdf}
		\label{fig:size}
		\subcaption{Number of rows and columns}
		\endminipage
		\caption{Clean files statistics}
		\label{fig:clean}
		\ref{fig:clean}
	\end{figure}
	
	The next operation is a cleaning step in order to get matrices (or dataframes) from the downloaded files. We manage to clean 14640 of the 55138 downloaded files (around 27\%). Some of them returned multiple subfiles like the ZIP files. Moreover, concerning the spreadsheets, we store one dataframe per sheet. In total, we get 23112 cleaned dataframes (for 72GB) coming from 9092 different pages. For the rest of the paper, when we refer to cleaned files, we mean these 23112 files. 
	
	Figure~\ref{fig:clean} allows us to visualize some statistics about this dataset. We notice that we clean a majority of GEOJSON and XML files: two semi-structured formats. \emph{Other} means the MIME type our algorithm detect is \emph{application/octet-stream}, a generic binary file without precise extension specified. Among the 651 we store, 382 come from an original ZIP file and 252 from an original shapefile. Most of them are For a matter of visualization, as with the left figure, we use a log scale. Most of them are produced by the departmental authorities named \emph{Direction Départementale des Territoires}. The right figure shows a scatter plot with the number of row and columns for each file. The original extension of files appears to have a influence on the result returned. Indeed, files with more rows are mostly originated from a CSV file (green stars). On the contrary, those with more columns are more often originated from spreadsheets (CDFV2 as blue triangles and Excel files as red crosses). However we observe that the two files with the more than 1000 columns come from CSV files. Some files appear to have only one column or one row. It is a imperfect result from the cleaning process. There are 1987 files with only one row, 1358 of them come from a GEOJSON files and 439 from an unidentified binary file (other). The files with only one column are 470 and 458 of them come from an unidentified binary file too. Again, both categories of files are mostly produced by departmental authorities \emph{Direction Départementale des Territoires} as the \emph{other} extension is prominent here.
	
	Among the 23112 clean files, 17147 (74\%) have not been reused, 898 (4\%) have been reused once and 4511 (20\%) twice. In \href{http://www.data.gouv.fr/fr/}{data.gouv.fr} 1711 reuses are listed, but only 38\% are concerned with our dataset (646 reuses).	
	
	Taking into account all the zipped files and potential subfiles from spreadsheets, we failed to clean 97703 elements. For 51744 of them (53\%), we can't even determine a specific extension and it return a generic binary MIME type (\emph{application/octet-stream} or \emph{other} in our study). The second largest proportion, 41551 files (43\%), is originally detected as a CSV file (or at least as a text file). Most of the errors are due to a failure while we try to extract a proper dataframe from the file (57495 failures), the rest is mostly due to the limits of \emph{chardet} library concerning the encoding inference ()\emph{UnicodeDecodeError}) or an impossibility to read a ZIP file (\emph{BadZipFile}). It is interesting to notice that two kinds of producer are overrepresented in the failures: the city of Lyon (\emph{Métropole de Lyon}) with 7411 errors and the well known departmental authorities \emph{Direction Départementale des Territoires}.
	
	\subsection{Metrics and cross-validation framework}
	
	The cross-validation framework starts once we compute the \ac{TFIDF} matrix. It allows us to validate the settings of both the topic extraction and the metric learning processes.
	
	%\begin{figure}
	%	\centering
	%	\includegraphics[width = .9\linewidth]{}
	%	\caption{JSON example}
	%	\label{fig:json-example}
	%\end{figure}
	
	\paragraph{Cross-validation}
	
	For each setting we define a random permutation cross-validation to generate 50 independent train/test datasets. We randomly permute our list of pages before splitting it in two equal parts (a train and a test datasets). Therefore, to be consistent with our reuse information (we collected at the page level), we don't have the same page present both in train and test datasets. Performing it 50 times per setting allows us get a better understanding of the setting consistency through samples variations and randomness.\\
	
	We start with a \ac{TFIDF} matrix. In order to optimize our file embedding, we first test the parameters relative to dimensionality reduction: the model for topic extraction (\ac{NMF} or \ac{SVD} algorithm) and the number of topics (20 different levels from 5 to 100). From the random permutation cross-validation result, we split the \ac{TFIDF} matrix in two. We use the train matrix to fit a topic extraction model. We obtain a topic space $W_{train}$, with a lower dimension, and a fitted dictionary we can apply to any other \ac{TFIDF} matrix to reduce its dimension. This is what we do on the \ac{TFIDF} test matrix to get a topic space $W_{test}$. 
	
	The second step consists in building X and y vectors from both topic spaces. We don't want to train a supervised classifier model, but only compute an \ac{AUC} with the Precision Recall Curve. This is achieved using \emph{$precision\_recall\_curve$} and \emph{$average\_precision\_score$} functions from \emph{sklearn} library. Considering two files, we need to feed the functions with a probability to reuse them together (X) and a 0-1 label reporting if they have actually been reused together or not (y). From the topic space, we randomly choose pairs of files and compute the multidimensional difference of their vectors. Finally, we use a specific norm (among L1, L2 and infinite norms) to get our score. Usually, a probability close to one predicts a label 1 (in our case the reuse), so to be consistent we take the opposite score. Two distant files in the topic space give a vector with large differences and thus a large score. Considering two distant files, the opposite score allows us to associate a very low (and negative) score to a probable label 0 and conversely. A negative score fits with \emph{sklearn} functions and can replace the probability in this case. We just need it to be increasing. To build $X_{train}$, $y_{train}$, $X_{test}$ and $y_{test}$, we randomly select pairs of files among train and test datasets. We control the ratio of reused pairs and arbitrarily set it to 30\%. We also set a maximum vector length to $90,000$ pairs in order to save computation time. 
	
	\paragraph{Precision Recall curve and AUC}
	
	The X and y vectors are used to compare settings to each other by computing an \ac{AUC}. Even if we don't specifically train a model on our pairs, because of our framework, we can validate our settings and compare our association score-label in the same way we would do in a classic binary classification task.
	The Precision-Recall curve shows precision-recall pairs for different probability thresholds, supposedly returned from a fitted classifier. In our case we replaced the probability by an increasing score. So it's equivalent to decide if two files have been reused based on their affinity score. If it is greater than the threshold, as with a simple binary classifier we \textit{predict} a reuse, otherwise no. By varying the decision threshold, several precision-recall pairs can be estimated such that
	
	\[
	precision = \frac{TP}{TP + FP}
	\]
	\[
	recall = \frac{TP}{TP + FN}
	\]
	
	with $TP$ the number of True Positives (reused pairs correctly predicted), $FP$ the number of False Positives (non reused pairs predicted as reused) and $FN$ the number of False Negatives (reused pairs badly predicted). In our case, the precision describes the ability of a supposed simple binary classifier not to falsely label as reused files that have not been reused together. On the other hand, the recall intuitively measures the ability of this supposed classifier to correctly predict all the reused pairs. Thus, the Precision-recall curve shows a trade-off between our propensity to predict relevant pairs as reused and our capacity to return all the truly reused pairs. Simply said, if the setting of our file embedding returns high recall but low precision, it doesn't discriminate enough the non reused files and tends to makes too many files close. On the opposite, if it returns high precision but low recall, it makes very few files close, but most of them are actually reused. The \emph{$average\_precision\_score$} then compute the \ac{AUC}, between 0 and 1. A high \ac{AUC} means a simultaneous high precision and recall: our file embedding presents many clusters which actually gathers reused files. By shuffling the labels we can even simulate random predictions. In this case, the \ac{AUC} is supposed to return the fraction of reused pairs, namely 0.3 (as we previously fixed it while building X).\\
		
	With such a framework, we can already test both topic extraction models for different number of topics. We can even switch the norm used to compute the affinity score from X and compare the different \ac{AUC} returned. A last parameter need to be tested: the choice to learn a new metric or not. We use the multidimensional differences from $X_{train}$ (before we compute a one-dimensional score with it) and $y_{train}$ in order to train a \ac{LSML} model. It returns a transformation matrix L we can directly apply to $W_{test}$ or $X_{test}$ before computing the affinity score to get $X_{test}^{ML}$.\\
	
	In practice, we parallelize this cross-validation framework at the topic extraction level to speed up the computation. We define 40 unique settings (20 different number of topics tested with \ac{NMF} and \ac{SVD} models). For each setting, a worker computes 50 times the full cross-validation pipeline. Each run returns 9 different \ac{AUC}. For the three possible norms (L1, L2 and infinite), the worker compute an regular \ac{AUC}, one based on random prediction and one with $X_{test}^{ML}$. With the parametrization related to topic extraction, we finally test 360 different settings using a cross-validation framework. For each of them, we collect 50 \ac{AUC} value in order to analyze its distribution and evaluate its efficiency.
	
	\subsection{Results}	
	
	After comparing 360 different settings, we can compare them and choose an optimal parametrization to build our file embedding from the full \ac{TFIDF} matrix. Some results are then presented, in relation to the topics themselves, the topic space and its ability to form pertinent clusters or associations between files.
	
	\subsubsection{Optimal parametrization}
	
	A first information is that almost all settings are performing better than a random prediction. Their lowest percentiles are generally above 0.3, the chance level. As we can see in appendix~\ref{fig:auc summary}, it seems that the best results in term of \ac{AUC} value are found for 20 topics or more. We fix the number of topics to 25 at first, in order to compare the models and the norms used. Then we search the optimal number of topics considering the best model and norm.\\
	
	\begin{figure}[]
		\minipage{0.48\textwidth}
		\includegraphics[width=\linewidth]{images/"boxplot auc models (nopage l2 25)".pdf}
		\label{fig:gridsearch-models}
		\subcaption{Models comparison}
		\endminipage\hfill
		\minipage{0.48\textwidth}
		\includegraphics[width=\linewidth]{images/"boxplot auc norms (nopage 25)".pdf}
		\label{fig:gridsearch-norms}
		\subcaption{Norms comparison}
		\endminipage
		\caption{AUC computed for norms and models comparison}
		\floatfoot{Note: Both figures are computed with 25 topics. On the left, figure a used a L2 norm. On the right, figure b shows result with a metric learned.}
		\label{fig:gridsearch}
		\ref{fig:gridsearch}
	\end{figure}

	According to figure~\ref{fig:gridsearch} every norm seems to have approximatively the same median \ac{AUC} around 0.45. This parameter does not impact so much our result. However, L1 and L2 norms clearly have higher minimum values for the \ac{AUC} around 0.32. We choose to use L2 norm for the rest of the study. Firstly, it presents a higher minimum \ac{AUC} than L1 norm when it's combined with a \ac{NMF} model. Secondly, it's consistent with the theoretical presentation of the Mahalanobis matrix section 4 and contrary to infinite and L1 norm, it's a smoother function. We can notice that every \ac{AUC} computed with the three norms perform at least just above the random prediction level.
	
	Concerning the best topic extraction model to choose, we set a \ac{NMF} model which present a slightly better $25^{th}$ percentile than \ac{SVD} model when then are combined with a learned metric. In this precise configuration, both models don't really differ from each other. However without metric learning, \ac{NMF} model can even present results worse than the chance. According to the overall distribution of the returned \ac{AUC}, we can notice the small but clear impact of metric learning as it improves the reuse recognition.
	
	\begin{figure}
		\centering
		\includegraphics[width = .5\linewidth, height = .5\linewidth]{images/"boxplot auc topics(nopage l2 NMF)".pdf}
		\caption{AUC computed for different number of topics}
		\floatfoot{Note: The L2 norm is used, along with NMF model and a learned metric.}
		\ref{fig:topics-comparison}
		\label{fig:topics-comparison}
	\end{figure}	
	
	Setting the L2 norm, the use of a \ac{NMF} model to build a first topic space and then its transformation by a Mahalanobis matrix, we compare the \ac{AUC} computed for different dimensionality reduction amplitudes. According to the figure~\ref{fig:topics-comparison} results start being interesting from 20 topics. With a smaller number of topics, we loose too much information to do better than chance. The best precision-recall pairs seems to be obtained with 25 topics. It return a median \ac{AUC} of 0.45 and a $75^{th}$ percentile just below 0.5. More topics still gives correct results until 100, but with no serious improvement. Moreover, a large number of topics reduces the interest of a dimensionality reduction. Our matrix $W$ is returned so sparse than some topics are quasi null. That is why we keep 25 topics to ensure the simplest, but yet efficient settings to build our file embedding.
	
	\subsubsection{Topics and Wordclouds}
	
	All our dataset is now used with the full \ac{TFIDF} matrix. We compute our file embedding with 25 topics through a \ac{NMF} algorithm. We also trained a Mahalanobis matrix to learn a metric and transform the topic space.
	
	\begin{figure}
		\centering
		\includegraphics[width = .5\linewidth, height = .5\linewidth]{images/"distance distribution violin plot".pdf}
		\caption{Cosine distance distribution in the best file embedding}
		\ref{fig:distance-distribution}
		\label{fig:distance-distribution}
	\end{figure}
	
	\paragraph{Topic space}
	
	In the figure~\ref{fig:distance-distribution}, we plot the distribution of distances within our file embedding. Contrary to the X affinity score and \ac{AUC} computation, use a cosine distance. We use the distance implemented in the \emph{scipy} library. The Cosine distance between two vectors u and v is defined by
	
	\[
	Cosine \: distance = 1 - \frac{u.v}{||u||_2||v||_2}
	\]
	
	where $u.v$ is the dot product between u and v. It measures the distance between two non-zero vectors, from $-1$ to $1$. If two vectors have the same direction, whatever magnitude they have, their cosine distance will be 0. The distance is 1 for two orthogonal vectors, a case illustrated for example by two files which do not share any topics. As the distance is independent of the magnitude of the vectors, it is definitely an orientation measure. Cosine distance perfectly fits our (25-dimensional) positive space returned from \ac{NMF} algorithm (and returns distance between 0 and 1). Thus, we interpret it as a measure of how much the subjects the two documents deal with are likely to be close. As it only considers the non-zero dimensions, this distance is still consistent with our sparse date. 
	
	The distribution of Cosine distance first shows that files from same extension do not appear to be closer in the embedding. It validates a part of our cleaning process where we return homogeneous dataframes from heterogeneous files with various format. In addition, we specify a list of stop words during in our text processing step to filter some repetitive and non informative words occurring only with specific extensions. For example this is the case with \textit{"featurecollection"}, \textit{"polygon"} or \textit{"geometry"} present in almost every GEOJSON files. Finally we build an embedding consistent with the different extension. Distance between two files seems to be independent of their extension. According to \emph{all} and \emph{other} groups, we can also say that in average files seem quite distant from each other. At least their vectorial representations have different directions within the topic space. An other feature with almost no impact on the distances between files is the share of one or many tags. These are key words provided by the metadata of \href{http://www.data.gouv.fr/fr/}{data.gouv.fr}.
	
	On the contrary, for significant proportion of files which come from the same pages or from the same producer we observe closer distances. The amplitude is larger with files sharing the same page. Finally, we the figure~\ref{fig:distance-distribution} shows a higher variance for distances between files reused together. Most of these pairs still present a relative closer distance than what we observe in the rest of the topic space.
	
	\paragraph{Wordclouds}
	
	An other way to analyze our topic space is to plot wordclouds using \emph{WordCloud} library. Considering our matrix factorization step $TFIDF = W.H$, this time we exploit the matrix $H$ which represents topics within words space. We transposed it and then for each columns (in other words, the topics) we sort the values. We keep the larger ones with their related word and plot one wordcloud per topic. The weight of every word in a topic determines its size in the plot. This is a graphical way to quickly determine which topics show a homogeneous and consistent meaning. Some of them could even be manually labeled. Moreover, we notice that this representation is independent of the metric learning step, as the matrix $H$ is not transformed by the Mahalanobis matrix. Lastly, our previous lemmatization process applied on our textual data, and more specifically in our stems, allows us to plot consistent wordclouds without root forms or redundancies.
	
	\begin{figure}[]
		\minipage{0.33\textwidth}
		\includegraphics[width=\linewidth]{images/"topic 5".pdf}
		\label{fig:wc-region}
		\subcaption{Region topic}
		\endminipage\hfill
		\minipage{0.33\textwidth}
		\includegraphics[width=\linewidth]{images/"topic 22".pdf}
		\label{fig:wc-police}
		\subcaption{Police and criminality topic}
		\endminipage\hfill
		\minipage{0.33\textwidth}
		\includegraphics[width=\linewidth]{images/"topic 17".pdf}
		\label{fig:wc-budget}
		\subcaption{Budget topic}
		\endminipage
		\caption{Three wordcloud topics}
		\label{fig:wc}
		\ref{fig:wc}
	\end{figure}

	We present three examples of homogeneous topics in the figure~\ref{fig:wc} \footnote{See appendix for the other topics}. The left one gathers quasi exclusively geographical words indicating a well known location or a French region. The second one probably comes from a group of police reports. It mostly includes vocabulary about crimes, thefts or burglaries. It is noteworthy that this is one of the most homogeneous and recurrent topics we get, even by reducing dimension to 20 or 30 topics. The third example, on the right, shows a mixed vocabulary with concepts related to budget (\textit{"budget"}, \textit{"gestion"} or \textit{"finance"}) and bureaucracy (\textit{"mission"} or \textit{"dgfip"} for \textit{Direction Générale des Finances publiques}) concepts.

	\begin{figure}[]
		\includegraphics[width=.5\linewidth]{images/"topic 21".pdf}
		\caption{Hydrography topic}
		\label{fig:wc water}
		\ref{fig:wc water}
	\end{figure}

	\begin{figure}[]
		\minipage{0.33\textwidth}
		\includegraphics[width=\linewidth]{images/"topic 21 (extension)".pdf}
		\label{fig:wc-extension}
		\subcaption{Extensions weights}
		\endminipage\hfill
		\minipage{0.33\textwidth}
		\includegraphics[width=\linewidth]{images/"topic 21 (producer)".pdf}
		\label{fig:wc-producer}
		\subcaption{Producers weights}
		\endminipage\hfill
		\minipage{0.33\textwidth}
		\includegraphics[width=\linewidth]{images/"topic 21 (tag)".pdf}
		\label{fig:wc-tag}
		\subcaption{Tags weights}
		\endminipage
		\caption{Sources of the water topic}
		\label{fig:wc water 2}
		\ref{fig:wc water 2}
	\end{figure}

	Most of the wordclouds illustrate a lexical field related to geographical information (land register, high risk area or urban planning). This is mainly due to the prominence of GEOJSON files in our dataset. The next example, from figure~\ref{fig:wc water}, is a topic dealing with hydrography terms (\textit{"etang"}, \textit{"calcaire"} or \textit{"ravine"}). To understand where do the topics come from, we plot additional wordclouds with extension, producer and tags information. We count these information for each file, then we weight this count by the value assigned to the file in matrix $W$ for each topic. It allows us to compute three matrices representing the weights of all extensions, producers and tags within the topics. Figure~\ref{fig:wc water 2} shows the three wordclouds returned for our hydrography topic. As we can observe on the left figure, most of the semantic information comes from GEOJSON files. We also learn that the main producer is \textit{Système d'information sur l'eau}. It is a public service which collects, stores and communicates data about water quality and its uses. So, it is not surprising to see the prominence of some tags like \textit{"environment"}, \textit{$"eaux\_de\_transition\_estuaires"$} or \textit{$"directive\_cadre\_eau"$}.
	
	\subsubsection{Reuse prediction}
	
	As we can see in appendices~\ref{fig:pca12}~\ref{fig:pca13} and~\ref{fig:pca23}, when we plot a \ac{PCA}, some pertinent information seem to be captured along the axes, at least for the three first components. To analyze what kind of files are clustered together in our file embedding we first need to determine such clusters. To define a neighborhood, we do not specify a maximum number of neighbors to identify, but a radius. Every files within the specified range is a neighbor of our target file. If we want to build a recommendation system to suggest potential documents to integrate with our target file, this neighborhood could be returned.
	
	\begin{figure}[]
		\minipage{0.48\textwidth}
		\includegraphics[width=\linewidth]{images/"precision recall curve (nmf nopage l2 ml 25)".pdf}
		\label{fig:prc-gridsearch}
		\subcaption{50 curves from the cross-validation step}
		\endminipage\hfill
		\minipage{0.48\textwidth}
		\includegraphics[width=\linewidth]{images/"precision recall curve".pdf}
		\label{fig:prc-best}
		\subcaption{Best file embedding}
		\endminipage
		\caption{Precision-recall curve}
		\floatfoot{Note: The red dot cross represents the balance chosen between precision and recall.}
		\label{fig:prc}
		\ref{fig:prc}
	\end{figure}
	
	\paragraph{Precision, recall and radius}
	
	If the radius is too small, the neighborhood will be too tight. The rare neighbors would be quasi similar to the target file. If the radius is too large, our neighborhood will capture a lot of noise. To optimize the radius, we use again the Precision-Recall curve. Plotted with all our data and the best setting (see figure~\ref{fig:prc}), we notice our embedding return an \ac{AUC} of 0.46 which is fair knowing that chance level is 0.3. This is exactly the median \ac{AUC} value we got from our 50 folds during the cross-validation step, using the exact same setting. Each precision-recall pair is related to a threshold. The latter discriminates between pairs of files potentially reusable or not, based on their distance in the embedding. As we compute our affinity score X between $-\infty$ and 0 (two closed files have a score equal to zero), thresholds have the same range. To compute a radius (real positive), we just take the opposite. In our case, a loose threshold means a low one and so a very large radius. This defines a huge neighborhood. If it is large enough, it includes all the actual reused files (so recall is 1) and precision tends to the actual ratio of reused files (0.3, according to the way we build X). In figure~\ref{fig:prc} we observe such results on the right of the curve. On the contrary, with a threshold and a radius close to zero, very few neighbors would be kept. In our figure, at the beginning of the curve, it shows a precision of 0.6 and a quasi null recall. We missed almost every potential reuses, but those we find are more often validated (neighbors are just next to our target file in the embedding). We want to be as close as possible of the top right corner to maximize our precision and our recall. The red cross represents our trade-off. With a precision of 0.5, half of the reuse we would predict with a simple binary classifier would be verified. Finally we get a recall greater than 0.6. In average, for every file we target, our algorithm can suggest a neighborhood including at least 60\% of files it has been reused with.
	
	\begin{figure}[]
		\includegraphics[width=.5\linewidth]{images/"boxplot neighbors".pdf}
		\caption{Neighbors distribution}
		\floatfoot{Note: Only the file with at least one neighbor are considered for this distribution.}
		\label{fig:neighbors}
		\ref{fig:neighbors}
	\end{figure}

	\paragraph{Neighborhood}
	
	Once we choose our best radius, we can fit a \emph{NearestNeighbors} model from \emph{sklearn}. The function \emph{radius\_neighbors} returns indices and distances of every files that lie within a ball of size radius around the targeted file. When we analyze the neighborhoods created, we notice than 7663 of the 23112 files have at least one neighbor. If we plot their distribution (see figure~\ref{fig:neighbors}), we observe a unbalanced shape. Most of the files with a neighborhood have less than 250 neighbors. On the top of the distribution, a group of files seems to share more than 2000 neighbors. Most of them are GEOJSON files. If we only consider the files with at least one neighbor, each neighborhood contains, in average, 143 neighbors and 69 reused pairs. For further analysis, we select some files with an interesting number of neighbors and reuses between these neighbors. 

	\begin{figure}[]
		\includegraphics[width=.8\linewidth]{images/"kneighbors pca 3d 20032".pdf}
		\caption{Potential files to reuse together}
		\floatfoot{Note: A 3-component PCA is applied on the neighborhood define around a file reporting births of horses. Orange relations represent an actual reuse. The orange area is the radius.}
		\label{fig:pca horse}
		\ref{fig:pca horse}
	\end{figure}
	
	The first example illustrates the case where the algorithm can suggest potential new reuses. Figure~\ref{fig:pca horse} represents a 3-components PCA performed on the neighborhood of the file \emph{Nombre de naissances d'équidés en France}. Components 1 and 3 are the most interesting. We observe six neighbors, two from the same page than our target file (red diamonds), three from the page \emph{Nombre de juments saillies en France} (blue crosses) and one from the page \emph{Périmètres des pays} (orange crosses). Both pages dealing with horse-riding data (red and blue points) come from the same producer, the French institute of horses and horse-riding. A reuse already exists with our target page about births of horses. However, this neighborhood also suggests us reusing this target page with data about mare mating. Such reuse seems relevant. 
	
	\begin{figure}[]
		\includegraphics[width=.8\linewidth]{images/"kneighbors pca 3d 20223".pdf}
		\caption{An already reused neighborhood}
		\floatfoot{Note: A 3-component PCA is applied on the neighborhood define around a French presidential election report.}
		\label{fig:pca election}
		\ref{fig:pca election}
	\end{figure}
	
	The second example illustrates the case where our neighborhood completely fits existing reuses. Figure~\ref{fig:pca election} shows a 3-components PCA on the neighborhood of the file \emph{Elections présidentielles 1965-2012}. We observe 54 neighbors with numerous reuses. Here again, the components 1 and 3 are the more relevant (the second component is even non-informative). The number of reuses if inflated by the fact that the original target page contains several files about French presidential election from 1965 to 2012. However, some reuses includes pages about French legislative election (for the same period). This neighborhood also suggest potential new reuses with data about 1992 referendum for example. Lastly, it includes what we could consider as less relevant: files dealing with driving licenses or emigration.

	\begin{figure}[]
		\includegraphics[width=.8\linewidth]{images/"kneighbors pca 3d 14360".pdf}
		\caption{A potential case of data augmentation}
		\floatfoot{Note: A 3-component PCA is applied on the neighborhood define around a \emph{Cours des Comptes} report.}
		\label{fig:pca compte}
		\ref{fig:pca compte}
	\end{figure}

	The third example illustrates the case where files come from the same producer and deal with the exact same topic. Figure~\ref{fig:pca compte} shows files from \emph{La Cours des Comptes}, a French administration which issues report about public finance. The target page is named \emph{Rapport d'observations définitives des chambres régionales et territoriales des comptes (2014)}, a budgetary report about regional administrations in 2014. The other reused cluster in the neighborhood is the same report, but for the year 2015. This neighborhood suggests us aggregating the same kind of data, but for different temporal periods. For a potential reuse, it could be a great opportunity to increase the number of sample. 
	
	\section{Discussion}
	
	This study is need to be read with a critical distance as it involves complex data, unusual methods and laborious problematic. But even so, it makes possible multiple improvements in different area.
	
	\subsection{Limitations and potential improvements}
	
	\paragraph{Data}
	
	According to the previous description of the pipeline used to build our dataset, from metadata collection to files cleaning, different kind of failures happened. We do not manage to download and clean all the data listed in \href{http://www.data.gouv.fr/fr/}{data.gouv.fr}. Our goal is simply to gather enough data to infer the most relevant statistical results. However, our current dataset presents some limitations. 
	
	Even if we can't correct the wrong URL during the downloading step, some solutions exist to store more data. It might be worth downloading some remote files. For example, the biggest platforms (as the National Institute for Statistics and Economic Studies) would deserve a specific code to collect their data. More generally, additional sources of open data could diversify and enlarge our dataset. This could be done with the World Bank or Eurostat. The latters would provide cleaned files with relevant socio-economic data. Lastly, two prominent and interesting sources of open data are the British and American version of \emph{data.gouv}.
	
	Concerning the cleaning step, it might be worth to focus on data produced by the city of Lyon and the recurrent departmental authorities \emph{Direction Départementale des Territoires}. Firstly they are the source of a majority of our cleaning errors. Secondly, even when it does not raise an exception, some of their files are badly cleaned with only one column or on row. Still, it will be difficult to get pass some limits as our cleaning capacity is mainly based on external library such that \emph{chardet} or \emph{magic}. If they fail to detect the encoding or the MIME type, our pipeline become inefficient.
	
	Besides limitations and potential improvements of our pipeline, some weaknesses are directly related to the data hosted by \href{http://www.data.gouv.fr/fr/}{data.gouv.fr}. Indeed, very few reuses are listed. There are 1711 reuses listed in the platform, but 1390 of them (81\%) concern only one page of data. This makes reuses less relevant to learn how to cross files from different origins.
	
	\paragraph{Methods}
	
	The use of \ac{NMF} algorithm to factorize matrix suffers from a interpretational drawback. According to Donoho and Stodden (2004), the decomposition returned by the algorithm may not be unique. Thus, the result would depend on the initialization and the starting values. If the \ac{TFIDF} evolves too much (by adding new files for example), the topics returned might loose their consistency. 
	
	Concerning the metric learning model, a limit can be defined too. As most of the supervised models, it suffers from unbalanced labels. This is clearly our situation as a very low proportion of pairs are actually reused (around 0.01\%). That is why, we make sure to keep 30\% of reused pairs when we build the vectors X and y. This is equivalent to perform an undersampling technique over non reused pairs. Such a method makes features appear with a higher variance than they do.
	
	Finally, we notice a potential improvement for the cross-validation framework. Besides the split at the page level to create train and test datasets, one could ensure that a reuse can't be present in both datasets. Formally, it would make the size control of each dataset more difficult. 
		
	\paragraph{Problematic}
	
	As we said, it might be worth having more reuses. With a majority of reuses concerning only one page fo data, the correlation between reuses and similarity increases. Our embedding learns that a reuse associates most of the time quasi identical files. It's relevant. However, this denies an interesting dimension of the reuse: crossing two files from completely different, but nonetheless complementary topics. More generally, define what could be a true reuse is subtle. For now, our embedding strongly depends on what has been actually done in \href{http://www.data.gouv.fr/fr/}{data.gouv.fr} in term of reuse. This is also our only way to validate if the suggestions returned by our algorithm are relevant or not. Therefore, we certainly miss some reuse opportunities. If a topic has never been reused we could not recognize the potential of related files.
	
	\subsection{Future work}
	
	With its results and its limits, this study gives us several prospects. Firstly, we can focus on the geographical data and especially on the GEOJSON extension. Using \emph{geopandas} library, we could efficiently parse and manipulate these files. As numerous geographical files are close from each other in huge cluster, being able to infer the geographical area concerned by their data would help us to discriminate them. Secondly, we could perform foreign key mining and record linkage over the closest files in the embedding. Ideally,It would be possible to automatically merge several files once we recognized a common id columns between them. Lastly, the relations and the limits described in this paper could help the platform \href{http://www.data.gouv.fr/fr/}{data.gouv.fr}. The pipeline and the suggestions of files returned could even be integrated to an user interface. 

	\section{Conclusion}	
	
	Integrate heterogeneous datasets can bring us to different paths. We choose to approach the problem with a statistical point of view and machine learning tools. 
	
	The first task is the building of a relevant dataset, big enough to ensure statistical results. It gives us the opportunity to develop algorithms in order to parse, clean and reshape numerous files on a large scale. The second task is the building of a file embedding, in order to easily compute distances between files and infer relevant relations. This involved unsupervised and supervised technique as matrix factorization or metric learning. A cross-validation pipeline with distinct train and test datasets allows us to control the results and validate any improvement. To fit with learning framework, we define a pretext task: the suggestion of reuses between files. 
	
	Finally, we present a study with multiple dimensions. Beyond reuse prediction, it is our capacity to infer relevant information about a file or a group of files that matters. Considering its successes and limits, this work illustrates the growing interest for data wrangling. Some results or problems raised could definitely interest future papers, as volume and diversity of data increase. 
	
	\bibliographystyle{abbrv}
	\bibliography{report_biblio}
	
	\section{Appendix}
	
	\begin{figure}[h]
		\centering
		\includegraphics[width = .6\linewidth]{images/"auc summary (nopage l2)".pdf}
		\caption{Cross-validation results}
		\label{fig:auc summary}
	\end{figure}

	\begin{figure}[h]
		\centering
		\includegraphics[width = .5\linewidth]{images/"acp_1_2".pdf}
		\caption{PCA - component 1 and 2}
		\label{fig:pca12}
	\end{figure}
	
	\begin{figure}[h]
		\centering
		\includegraphics[width = .5\linewidth]{images/"acp_1_3".pdf}
		\caption{PCA - component 1 and 3}
		\label{fig:pca13}
	\end{figure}
	
	\begin{figure}[h]
		\centering
		\includegraphics[width = .5\linewidth]{images/"acp_2_3".pdf}
		\caption{PCA - component 2 and 3}
		\label{fig:pca23}
	\end{figure}
	
	\begin{figure}[h]
		\minipage{0.33\textwidth}
		\includegraphics[width=\linewidth]{images/"topic 0".pdf}
		\subcaption{Topic 0}
		\endminipage\hfill
		\minipage{0.33\textwidth}
		\includegraphics[width=\linewidth]{images/"topic 1".pdf}
		\subcaption{Topic 1}
		\endminipage\hfill
		\minipage{0.33\textwidth}
		\includegraphics[width=\linewidth]{images/"topic 2".pdf}
		\subcaption{Topic 2}
		\endminipage
	\end{figure}

	\begin{figure}[h]
		\minipage{0.33\textwidth}
		\includegraphics[width=\linewidth]{images/"topic 3".pdf}
		\subcaption{Topic 3}
		\endminipage\hfill
		\minipage{0.33\textwidth}
		\includegraphics[width=\linewidth]{images/"topic 4".pdf}
		\subcaption{Topic 4}
		\endminipage\hfill
		\minipage{0.33\textwidth}
		\includegraphics[width=\linewidth]{images/"topic 6".pdf}
		\subcaption{Topic 6}
		\endminipage
	\end{figure}

	\begin{figure}[h]
		\minipage{0.33\textwidth}
		\includegraphics[width=\linewidth]{images/"topic 7".pdf}
		\subcaption{Topic 7}
		\endminipage\hfill
		\minipage{0.33\textwidth}
		\includegraphics[width=\linewidth]{images/"topic 8".pdf}
		\subcaption{Topic 8}
		\endminipage\hfill
		\minipage{0.33\textwidth}
		\includegraphics[width=\linewidth]{images/"topic 9".pdf}
		\subcaption{Topic 9}
		\endminipage
	\end{figure}

	\begin{figure}[h]
		\minipage{0.33\textwidth}
		\includegraphics[width=\linewidth]{images/"topic 10".pdf}
		\subcaption{Topic 10}
		\endminipage\hfill
		\minipage{0.33\textwidth}
		\includegraphics[width=\linewidth]{images/"topic 11".pdf}
		\subcaption{Topic 11}
		\endminipage\hfill
		\minipage{0.33\textwidth}
		\includegraphics[width=\linewidth]{images/"topic 12".pdf}
		\subcaption{Topic 12}
		\endminipage
	\end{figure}

	\begin{figure}[h]
		\minipage{0.33\textwidth}
		\includegraphics[width=\linewidth]{images/"topic 13".pdf}
		\subcaption{Topic 13}
		\endminipage\hfill
		\minipage{0.33\textwidth}
		\includegraphics[width=\linewidth]{images/"topic 14".pdf}
		\subcaption{Topic 14}
		\endminipage\hfill
		\minipage{0.33\textwidth}
		\includegraphics[width=\linewidth]{images/"topic 15".pdf}
		\subcaption{Topic 15}
		\endminipage
	\end{figure}

	\begin{figure}[h]
		\minipage{0.33\textwidth}
		\includegraphics[width=\linewidth]{images/"topic 16".pdf}
		\subcaption{Topic 16}
		\endminipage\hfill
		\minipage{0.33\textwidth}
		\includegraphics[width=\linewidth]{images/"topic 18".pdf}
		\subcaption{Topic 18}
		\endminipage\hfill
		\minipage{0.33\textwidth}
		\includegraphics[width=\linewidth]{images/"topic 19".pdf}
		\subcaption{Topic 19}
		\endminipage
	\end{figure}

	\begin{figure}[h]
		\minipage{0.33\textwidth}
		\includegraphics[width=\linewidth]{images/"topic 20".pdf}
		\subcaption{Topic 20}
		\endminipage\hfill
		\minipage{0.33\textwidth}
		\includegraphics[width=\linewidth]{images/"topic 23".pdf}
		\subcaption{Topic 23}
		\endminipage\hfill
		\minipage{0.33\textwidth}
		\includegraphics[width=\linewidth]{images/"topic 24".pdf}
		\subcaption{Topic 24}
		\endminipage
	\end{figure}
	
\end{document}